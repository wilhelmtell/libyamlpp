\documentclass{article}

\usepackage{pxfonts}

\author{Alex Vall\'ee \and Matan Nassau}
\title{COMP 446 Team Project Report:  libyaml++}

\begin{document}
\maketitle

\section{Outline}
The report (5--20 pages) should include some or all of the following:

\begin{enumerate}
  \item An overview of your project
  \item Your motivation for doing this project
  \item Problems encountered and how they were solved (or not solved)
  \item Instructions for users (if necessary)
\end{enumerate}

\section{Evaluation}
\frame{\textbf{TODO:}  wipe off this section}
\vspace{2em}

\begin{tabular}{rp{40ex}r}
  \hline
  \textbf{Deliverable} & \textbf{Details} & \textbf{Points} \\
  \hline
  \textbf{Source Code} & Algorithms (standard, STL, or your own) & 5 \\
  & Data Structures (standard, STL, or your own) & 5 \\
  & Genericity (inheritance or templates & 5 \\
  & Performance (program works as claimed, does not fail) & 5 \\
  & Libraries (use of STL and/or others) & 10 \\
  & Structure (indentation, spacing, file organization, comments) & 10 \\
  \textbf{Documentation} & User Manual (clear, concise instructions about using the program) & 5 \\
  & Implementation Manual (design, architecture, modue descriptions; can use Doxygen) & 5 \\
  & Individual Experience Report (what you did; what you learned; problems found and solved) & 10 \\
  \hline
\end{tabular}
\section{Overview}
Libyaml++ is a YAML parsing library for C++. It aims to create a simple way to
read and write yaml document streams. Users of the library should have to
remember few functions from the library in order to make effective use of it.

\section{Motivation}
With experience from classes such as SOEN229 and COMP335 from the software
engineering program, we decided we wanted to practice our parsing skills.

YAML is becoming popular in applications such as log and configuration files.
With this knowledge, we searched for other implementations, but none
specifically for C++.


%
% Problems encountered go here.
%

%
% Instructions here.
%


\end{document}
