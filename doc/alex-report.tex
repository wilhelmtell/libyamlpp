\documentclass{article}

\usepackage{setspace}

\usepackage{pxfonts}
\usepackage[usenames,dvipsnames]{color}
\usepackage[colorlinks=true,linkcolor=black,filecolor=black,citecolor=black,urlcolor=black]{hyperref}

\author{Alex Vall\'ee\\5713315}
\title{COMP 446 Team Project Individual Experience Report:\\Alex Vall\'ee}

\begin{document}
\maketitle
\doublespacing

\section{Things Learned from the Project}

During the development of the project I had chances to learn a few new things.
Obviously I learned about YAML and its specification, and some applications of
C++ topics.

\begin{description}
  \item[Composite Pattern] As a pattern I've heard in both COMP446 and SOEN343
    this semester, I explored the possibility of exporting our expression tree
    using this pattern.  My specific implementation wasn't very good for the
    problem at hand, but I did learn how to implement the pattern in C++.
  \item[Idiomatic C++] In coding the various parts of the library, and
    specifically exploring data structures with pairs of iterators.
\end{description}

\section{My expectations}
For the project I was expecting to practice the use of build tools, and
practicing the creation of template functions and classes. But unfortunately,
our design did not include the created of templated functions or classes.
Matan wrote the original build scripts for waf, then bjam, and finally the
multipurpose shell script.  I was also expecting to practice communication, use
of \emph{git} the distributed version control system, and use of \LaTeX, in
which I hadn't used much recently.  I was satisfied with these last 3 points.

\section{Feelings about C++}

I first had interest in the language around the age of 16.  A popular game I
enjoyed at the time had the possibility of being modified by the player, by
modifying the gameplay library whose source code was available for download.  I
became interested in the language but never did any programming with it. Once I
started C\'EGEP, I had the opportunity to take an introductory course, however
the assignment were very plain and not very challenging.  With more computer
science training however I now think of programming problem in terms of a C++
solution first.

I enjoy the power and versatility of the language.  I have a great appreciation
for the language's idioms.  I also very much enjoy the clever tricks devised
such as the overloading of the \verb|<<| operator for output (although I
dislike the syntax) and template metaprogramming, though mostly just for the
novelty factor.

I have reservation about the language when development speed is a factor. C++
can we quite verbose in comparison to other languages, especially fourth
generation ones.  I believe that such a language would be much better for
prototyping a real-world project to demonstrate to the client quickly and then
reimplement in C++ once a working specification is drafted from the
requirements gathered from the prototype. Notably, for the SOEN341 project last
year, we were not able to finish the project on time mostly due to the large
amounts of code we needed to write just in order to read in our data files.

\section{Feelings about \textit{COMP446}}
I believe this class should remain a 400-level but I would like to be more
advanced.  Most student in the class did not seem very passionate, thus all the
syntax and usage needs to be taught to them.  Computer science student should
be able to pick up the basic concepts of a new programming language relatively
easily, so that a 400-level C++ class can concentrate much more on advanced
techniques and concepts, instead of having to start from the basics of what it
a literal, what is a variable, and so on...

I was looking forward to taking a class with Dr. Grogono.  He was a good
present, and provided prompt feedback on assignment, midterms, and inquiries.
If I have the chance, I would much prefer to take another course with this
professor over many of the other professors in Concordia's Computer Science and
Software Engineering department.  Most have trouble with language,
communication, and just plain mean.

Thanks.


\end{document}
